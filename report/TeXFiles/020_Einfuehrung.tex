\chapter{Einfuehrung}
\label{ch:Einfuehrung}

\noindent
This project aims to build a real-time traffic simulation application using Java as the main programming language and its various libraries, such as JavaFX, TraCI as a Service (TraaS), etc. It also utilizes functionalities of the Simulation of Urban Mobility (SUMO) software package to visualize traffic network and traffic flows. The application is designed to provide some basic mobility control functionalities and allows for user's interactive control, enabling its use for teaching, learning, and researching urban mobility.
\medskip

\noindent
The concentration of the project is a wrapper runner application that calls functions and accesses objects from external libraries to simulate real-time traffic, resembling a bird's eye view of a complex apparatus with independent components. This is the essence of object-oriented programming (OOP) design. 
\medskip

\noindent
Working development and documentation of the project can be found in our GitHub repository: 

\url{https://github.com/hunggiadao/realtime-traffic-simulation-java-oop}
\medskip


% ...



%Am Ende des Einführungskapitels ein kurzer Überblick über die kommenden Kapitel in einem Absatz (Beschreibung der Struktur der Thesis):

%...
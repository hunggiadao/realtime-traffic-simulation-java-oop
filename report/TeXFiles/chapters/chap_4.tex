\chapter*{Conclusion and Future Work}
\addcontentsline{toc}{chapter}{Conclusion and Future Work}

\section{Conclusion}
\noindent
By Milestone~2, we delivered a functional prototype of the real-time traffic simulation application. The system can connect to a live SUMO simulation via TraCI, run or step the simulation, and visualize the network and moving vehicles directly in the JavaFX GUI. Core interactive features—vehicle injection and manual traffic light control—are implemented, enabling users to actively influence traffic flow during runtime.
\medskip

\noindent
The main challenges identified at this stage are performance and update-frequency trade-offs when handling large vehicle counts, particularly during stress testing. These findings guide the next development steps: improving rendering efficiency, reducing unnecessary UI refresh work, and extending the control logic. Overall, Milestone~2 establishes a solid, end-to-end pipeline from SUMO to a responsive GUI with tangible control capabilities.

\section{Future Work}
\noindent
Building on the functional prototype, the final phase of development will focus on optimization, advanced features, and user experience refinements.

\subsection{Performance Optimization}
\noindent
To address the bottlenecks identified during stress testing, we plan to:
\begin{itemize}
    \item \textbf{Optimize Map Rendering:} Implement spatial partitioning (e.g., QuadTree) to only render vehicles and map elements currently within the viewport.
    \item \textbf{Throttle UI Updates:} Decouple the data fetch rate from the render rate to ensure the UI remains responsive even when the simulation is running at high speeds.
\end{itemize}

\subsection{Advanced Traffic Control}
\noindent
We will extend the current manual control capabilities with automated logic:
\begin{itemize}
    \item \textbf{Automated Traffic Light Management:} Implement algorithms that dynamically adjust signal phases based on real-time queue lengths detected by the simulation.
    \item \textbf{Vehicle Grouping:} Allow users to define groups of vehicles and apply batch commands (e.g., "reroute all buses").
\end{itemize}

\subsection{Reporting and Analytics}
\noindent
The final application will include comprehensive reporting tools:
\begin{itemize}
    \item \textbf{Data Export:} Functionality to export simulation statistics (travel times, congestion levels) to CSV or PDF formats.
    \item \textbf{Enhanced Dashboard:} A more detailed statistics view including average waiting times and CO2 emission estimates.
\end{itemize}
\medskip

\noindent
\begin{figure}[!ht]
    \centering
    \includegraphics[width=0.9\textwidth]{../assets/main_ui_milestone_2.png}
    \caption{Proposed layout for the advanced analytics dashboard. Note the 'Export' button (top right) and the expandable 'Charts' pane, which will be fully implemented in Milestone 3.}
    \label{fig:future-dashboard}
\end{figure}

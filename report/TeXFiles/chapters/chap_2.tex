\chapter{Project Workplan}
\label{ch:ProjectWorkplan}

%%%%%%%%%%%%%
\section{Timeline}
\subsection{Milestone 2 Status (Due: 14.12.2025)}

\noindent
Milestone~2 focused on delivering a functional prototype with core features implemented. The current status is summarized below.
\medskip

\begin{itemize}
    \item \textbf{Working Application:} Live SUMO connection, map visualization inside the GUI, vehicle injection, and traffic light control are implemented.
    \item \textbf{Code Documentation:} Core classes contain Javadoc and inline comments (e.g., connection, vehicle, and map rendering components).
    \item \textbf{User Guide Draft:} A draft user guide is provided (\texttt{userguide.md}) describing setup and basic usage.
    \item \textbf{Test Scenario:} A stress scenario configuration (\texttt{Stress.sumocfg}) is provided to test high vehicle counts.
    \item \textbf{Continuous Integration:} A CI pipeline was established to automate build and test processes, ensuring code stability.
    \item \textbf{Progress Summary:} Performance and UI-update frequency were identified as key bottlenecks under heavy load.
\end{itemize}

\subsection{Timeline for Final Application}
\begin{itemize}
    \item \textbf{Full GUI Implementation:} Complete map visualization, controls, and statistics dashboard by 5th January
    \item \textbf{Vehicle Grouping/Filtering:} Implement filtering and categorization features by 31st December
    \item \textbf{Traffic Light Adaptation:} Manual and rule-based control system by 31st December
    \item \textbf{Exportable Reports:} CSV and PDF export functionality by 31st December
    \item \textbf{Final Documentation:} Updated user guide and comprehensive code documentation by 10th January
    \item \textbf{Project Summary:} Enhancements overview and design decisions report by 10th January
\end{itemize}

%%%%%%%%%%%%%
\section{Task Distribution}

\noindent
Team roles were revisited and refined during Milestone~2 to match the implemented components.
\medskip

\begin{itemize}
    \item \textbf{Traffic Network Files:} Raees Ashraf Shah
    \item \textbf{TraCI Connector + SUMO Integration:} Gia Hung Dao
    \item \textbf{Traffic Light Control (wrapper + UI):} Gia Hung Dao, Raees Ashraf Shah
    \item \textbf{Vehicle Wrapper + Vehicle Injection:} Khac Uy Pham, Gia Hung Dao
    \item \textbf{GUI Implementation (JavaFX):} Huu Trung Son Dang
    \item \textbf{Infrastructure Wrapper Classes:} Huy Hoang Bui
    \item \textbf{Logger + Run Scripts:} Huu Trung Son Dang
    \item \textbf{Architecture and UML Diagrams:} Khac Uy Pham
\end{itemize}

\section{Challenges and Risks}

\begin{itemize}
    \item \textbf{Performance under load:} With many vehicles, frequent UI refreshes (map + table) can become CPU-intensive.
    \item \textbf{Update frequency trade-offs:} Increasing simulation step length or throttling table/map updates reduces load but lowers UI responsiveness.
    \item \textbf{Robust TraCI lifecycle:} SUMO may terminate or close the socket when a scenario ends; the application must handle disconnects gracefully.
\end{itemize}

\section{Milestone 2 Progress Summary}

\noindent
Milestone~2 focused on converting the Milestone~1 UI into a functional prototype with end-to-end SUMO integration. The following summary highlights the status and main outcomes.
\medskip

\begin{itemize}
    \item \textbf{Status:} Core prototype features completed (live SUMO connection, in-app map rendering with moving vehicles, vehicle injection, and traffic light control).
    \item \textbf{Key changes:} Added a real-time stepping loop, integrated vehicle state retrieval for the table and map, added injection controls (count/speed/color), and implemented traffic light phase switching and duration adjustments.
    \item \textbf{Stress testing:} Used a dedicated stress configuration (\texttt{Stress.sumocfg}) to validate behavior at higher vehicle counts and to observe performance bottlenecks.
    \item \textbf{Challenges:} Under load, UI updates (map redraw + table refresh) can be expensive; practical mitigation is to reduce refresh frequency or increase step length.
\end{itemize}

\subsection{Software Engineering Practices (Git)}

\begin{itemize}
    \item \textbf{Branching + PRs:} Development was performed on feature branches and merged via pull requests (e.g., simulation classes, traffic light wrapper updates, UI fixes).
    \item \textbf{Commit messages:} Commits generally use action-oriented prefixes (e.g., \textit{feat}, \textit{fix}, \textit{update}) and describe the affected subsystem (UI, simulation, wrappers, build).
    \item \textbf{Traceability:} Milestone~2 work is visible through incremental commits that introduce the simulator loop and vehicle-management abstractions, followed by bug fixes and UI integration improvements.
\end{itemize}

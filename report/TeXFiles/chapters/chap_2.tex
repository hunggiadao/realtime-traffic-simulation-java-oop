\chapter{Project Workplan}
\label{ch:ProjectWorkplan}

%%%%%%%%%%%%%
\section{Timeline}
\subsection{Final Milestone Status}

\noindent
The final milestone focused on delivering a fully fletched application with fully functional features. In addition, the application must run without non-reproducible errors. The application also logs all runtime information and saves it into a log file for later inspection. The status is summarized below.
\medskip

\begin{itemize}
    \item \textbf{Final Application}: Full GUI with map, controls, statistics; Vehicle grouping/filtering; Traffic light adaptation (manual or rule-based); Exportable reports (CSV and/or PDF).
    
    \item \textbf{Final Documentation}: Updated user guide; Final code documentation; Summary of enhancements and design decisions.
    
    \item \textbf{Presentation}: Live demo; Architecture and feature explanation o Performance evaluation; Team reflection.
    
    \item \textbf{Test Scenario:} A stress scenario configuration (\texttt{Stress.sumocfg}) is provided to test high vehicle counts.
    
    \item \textbf{Final Retrospective}
    
    \item \textbf{Clean Git Repository}
    
    \item \textbf{Sample Exported Reports}
    
    \item \textbf{Progress Summary}: Performance and UI-update frequency were identified as key bottlenecks under heavy load.
\end{itemize}

\subsection{Timeline for Final Application}
\begin{itemize}
    \item \textbf{Full GUI Implementation}: Complete map visualization, controls, and statistics dashboard by 5th January
    
    \item \textbf{Vehicle Grouping/Filtering}: Implement filtering and categorization features by 5th January
    
    \item \textbf{Traffic Light Adaptation}: Manual and rule-based control system by 8th January
    
    \item \textbf{Exportable Reports}: CSV and PDF export functionality by 10th January
    
    \item \textbf{Final Documentation}: Updated user guide and comprehensive code documentation by 18th January
    
    \item \textbf{Project Summary}: Enhancements overview and design decisions report by 18th January
    
    \item \textbf{Presentation}:
    	\subitem Live demo 19th January
    	
    	\subitem Architecture and feature explanation by 18th January
    	
    	\subitem Performance evaluation by 12th January
    	
    	\subitem Team reflection by 18th January
    	
    \item \textbf{Final Retrospective} by 18th January
    
    \item \textbf{Clean Git Repository} by 18th January
    
    \item \textbf{Sample Exported Reports} by 18th January
\end{itemize}

%%%%%%%%%%%%%
\section{Task Distribution}

\noindent
Team roles for the entire project are as follows:
\medskip

\begin{itemize}
    \item \textbf{Traffic Network Files}: Raees Ashraf Shah
    
    \item \textbf{TraCI Connector + SUMO Integration}: Gia Hung Dao
    
    \item \textbf{Traffic Light Control (wrapper + UI)}: Gia Hung Dao, Raees Ashraf Shah
    
    \item \textbf{Hotkey Functionality}: Raees Ashraf Shah
    
    \item \textbf{Vehicle Wrapper + Vehicle Injection}: Huu Trung Son Dang
    
    \item \textbf{GUI Implementation (JavaFX)}: Huu Trung Son Dang
    
    \item \textbf{Infrastructure Wrapper Classes}: Huy Hoang Bui, Gia Hung Dao
    
    \item \textbf{Logger + Run Scripts}: Huu Trung Son Dang
    
    \item \textbf{Exception handler}: Huu Trung Son Dang
    
    \item \textbf{CSV and PDF Exporter}: Raees Ashraf Shah, Gia Hung Dao
    
    \item \textbf{Architecture and UML Diagrams}: Khac Uy Pham, Huu Trung Son Dang
\end{itemize}

\section{Challenges and Risks}

\begin{itemize}
    \item \textbf{Performance under load}: With many vehicles, frequent UI refreshes (map + table) can become CPU-intensive.
    
    \item Some network-wide function calls used to gather entire information take longer than do function calls with specific Object ID.
    
    \item \textbf{Update frequency trade-offs}: Increasing simulation step length or throttling table/map updates reduces load but lowers UI responsiveness.
    
    \item \textbf{Robust TraCI lifecycle}: SUMO may terminate or close the socket when a scenario ends; the application must handle disconnects gracefully.
    
    \item Some metric functions in \texttt{EdgeWrapper} class took longer than other functions, since their runtime complexity grows exponentially with the number of edges and vehicles.
\end{itemize}

\section{Project Progress Summary}

\noindent
The final milestone focused on finishing ongoing features from Milestone 2, adding final features, and fixing all runtime bugs. The following summary highlights the status and main outcomes.
\medskip

\begin{itemize}
    \item \textbf{Status}: All features completed (live SUMO connection, in-app map rendering with moving vehicles, vehicle injection, filtering views, traffic light control, hotkeys, statistics recording, and file exports).
    
    \item \textbf{Key changes}: integrated vehicle state and edge state retrieval for the table and map, added traffic light phase control, added metrics charts and diagrams, implemented CSV and PDF export features.
    
    \item \textbf{Stress testing}: Used a dedicated stress configuration (\texttt{Stress.sumocfg}) to validate behavior at higher vehicle counts and to observe performance bottlenecks.
    
    \item \textbf{Challenges}: Under load, UI updates (map redraw + table refresh) can be expensive; practical mitigation is to reduce refresh frequency or increase \texttt{step-length}.
\end{itemize}

\section{Software Engineering Practices (Git)}

\begin{itemize}
    \item \textbf{Branching + Pull Requests}: The main branch is push protected. Development must be performed on feature branches and merged into main via pull requests (e.g., simulation classes, traffic light wrapper updates, UI fixes).
    
    \item \textbf{Commit messages}: Commits generally use action-oriented prefixes (e.g., \textit{feat}, \textit{fix}, \textit{update}, \textit{add}, \textit{remove}, etc.) and describe the affected subsystem (UI, simulation, wrappers, build).
    
    \item \textbf{Traceability}: the project work is visibly traceable through incremental commits that introduce the simulator loop and vehicle-management abstractions, followed by bug fixes and UI integration improvements. Commit history can be viewed in the Github repository.
\end{itemize}

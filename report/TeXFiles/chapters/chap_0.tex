\chapter*{Abstract}
\addcontentsline{toc}{chapter}{Abstract}

\noindent
This project develops a real-time traffic simulation application using Java, JavaFX, and the Simulation of Urban Mobility (SUMO). The goal is to provide an interactive desktop GUI that connects to a running SUMO simulation via TraCI (TraaS) and enables users to visualize and control key traffic elements.
\medskip

\noindent
By the final milestone, the project has been fully implemented with completely functional features with few to no runtime errors. The application can establish a live SUMO connection, advance the simulation in a continuous manner (or step-by-step), and render the selected SUMO network directly inside the JavaFX GUI with moving vehicles. The animation system is fine-tuned to render granular changes in movement, with 20 granular images to render for every time step (step-length of 50 ms). It also supports interactive vehicle injection (selecting a target edge, number of vehicles, speed, and color) and manual traffic light control (switching phases and adjusting phase duration). Furthermore, hotkey functionality has been incorporated to enable swift state changes without the need for cursor interaction. For usability and debugging, the GUI includes live status indicators, a vehicle table and chart, and optional filters (e.g., by vehicle color, speed, and congestion heuristics).
\medskip

\noindent
In addition to the working application, the final deliverables include inline code documentation (Javadoc), a complete user guide, and a stress-test scenario (\texttt{Stress.sumocfg}) to evaluate performance with a large number of vehicles. The report summarizes implemented features, current limitations, and the main challenges discovered during stress testing. All available resources can be accessed from the project’s Github repository. Details are provided below.
\medskip
